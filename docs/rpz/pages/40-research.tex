\section{Экспериментальная часть}

В данном разделе проведён сравнительный анализ реализаций сервиса поиска событий с использованием PostgreSQL и ElasticSearch

\subsection{Технические характеристики}

Замеры времени выполнялись на личном ноутбуке. Технические характеристики устройства, на котором выполнялось тестирование представлены далее:

\begin{itemize}[label=---]
	\item операционная система Windows 10 Домашняя;
	\item память 16 Гбайт;
	\item процессор 3.20 ГГц 4‑ядерный процессор Intel Core i5 11-го поколения;
	\item процессор имеет 4 физических и 8 логических ядер.
\end{itemize}

Во время замеров ноутбук был включен в сеть электропитания.

\subsection{Время выполнения алгоритмов}

Для замера времени работы алгоритмов использовалась функция \linebreak time.Now() из стандартной библиотеки на Golang.

Для проведения замеров времени поиска с фильтрацией и сортировкой использовался запрос c параметрами, приведенными на листинге \ref{lst:req-f}, а параметры запроса для замеров времени поиска без фильтрации и сортировки приведен на листинге \ref{lst:req-wf}. В обоих приведенных запросах X -- задаваемое количество запрашиваемых записей.

\begin{center}
	\captionsetup{justification=raggedright,singlelinecheck=off}
	\begin{lstlisting}[label=lst:req-f,caption=Параметры запроса с указанием фильтрациии и сортировки]
{
	"paginate": {
		"Page": 1,
		"PageSize": X
	},
	"filter":{
		"FTSearchStr": {
			"Str": "My"
		},
		"CreatorLogin": {
			"Ts": "Us"
		},
		"Description": {
			"Ts": "Desc"
		},
		"TagName": {
			"Ts": "Teg"
		}
	},
	"sort": {
		"CreatorLogin": "ASC"
	}
}
	\end{lstlisting}
\end{center}

\begin{center}
	\captionsetup{justification=raggedright,singlelinecheck=off}
	\begin{lstlisting}[label=lst:req-wf,caption=Параметры запроса без указания фильтрациии и сортировки]
{
	"paginate": {
		"Page": 1,
		"PageSize": X
	},
}
	\end{lstlisting}
\end{center}

Результаты замеров времени работы реализаций сервиса поиска с использованием PostgreSQL и ElasticSearch приведены в таблице \ref{tbl:time}. На рисунках \ref{fig:time-db} и \ref{fig:time-es} приведена графическая интерпретация замеров времени. В первом столбце таблиц указывается количество запрашиваемых записей, а во втором и третьем --- время поиска с фильтрацией и без фильтрации соответственно. Замеры производились по 500 раз для каждого количества запрашиваемых записей. На момент осуществления замеров в таблице Postgres events было 100000 записей. Столько же документов были и в индексе ElasticSearch events.

\begin{table}[h]
	\centering
	\captionsetup{justification=raggedright,singlelinecheck=off}
	\caption{Результаты замеров времени поиска с использованием PostgreSQL}
	\label{tbl:time}
	\begin{tabular}{|c|c|c|}
		\hline
		Кол-во записей & С фильтрами, млс & Без фильтров, млс \\ \hline
		      10       &    13.833706    &    11.289715     \\ \hline
		      20       &    14.250382    &    11.753059     \\ \hline
		      50       &    14.960792    &    11.462349     \\ \hline
		     100       &    16.769814    &    12.062963     \\ \hline
		     200       &    20.102638    &    14.037806     \\ \hline
		     300       &    21.950034    &    16.361748     \\ \hline
		     400       &    28.771323    &    23.641660     \\ \hline
	\end{tabular}
\end{table}

\begin{table}[h]
	\centering
	\captionsetup{justification=raggedright,singlelinecheck=off}
	\caption{Результаты замеров времени поиска с использованием ElasticSearch}
	\label{tbl:time}
	\begin{tabular}{|c|c|c|}
		\hline
		Кол-во записей & С фильтрами, млс & Без фильтров, млс \\ \hline
		10       &    67.975325   &     51.430142    \\ \hline
		20       &    69.194931   &     53.183822    \\ \hline
		50       &    72.861184   &     54.079701    \\ \hline
		100       &   77.778041    &    57.135915     \\ \hline
		200       &   86.193629    &    64.008997     \\ \hline
		300       &   93.549356    &    68.152554     \\ \hline
		400       &   98.196287    &    71.863563     \\ \hline
	\end{tabular}
\end{table}



\begin{figure}[h]
	\begin{center}
		\begin{tikzpicture}
			\begin{axis}[
				legend pos = north west,
				xlabel=Количество запрашиваемых записей,
				ylabel=Время в миксосекундах,
				minor tick num = 1,
				grid = both,
				major grid style = {lightgray},
				minor grid style = {lightgray!25},
				%xtick distance = 50,
				width = 0.9\textwidth,
				height = 0.55\textwidth]
				
				\addplot[
				blue,
				semithick,
				mark = x,
				mark size = 3pt,
				thick,
				] file {assets/data/db.tsv};
				
				\addplot[
				green,
				semithick,
				mark = x,
				mark size = 3pt,
				thick,
				] file {assets/data/db_filters.tsv};
				
				\legend{Поиск без фильтрации и сортировки, 	Поиск с фильтрацией и сортировкой}
			\end{axis}
		\end{tikzpicture}
	\end{center}
	\caption{Зависимость времени выполнения поиска в Postgres в зависимости от запрашиваемого количества записей}
	\label{fig:time-db}
\end{figure}


\begin{figure}[h]
	\begin{center}
		\begin{tikzpicture}
			\begin{axis}[
				legend pos = north west,
				xlabel=Количество запрашиваемых записей,
				ylabel=Время в миксосекундах,
				minor tick num = 1,
				grid = both,
				major grid style = {lightgray},
				minor grid style = {lightgray!25},
				%xtick distance = 50,
				width = 0.9\textwidth,
				height = 0.55\textwidth]
				
				\addplot[
				blue,
				semithick,
				mark = x,
				mark size = 3pt,
				thick,
				] file {assets/data/elastic.tsv};
				
				\addplot[
				green,
				semithick,
				mark = x,
				mark size = 3pt,
				thick,
				] file {assets/data/elastic_filters.tsv};
				
				\legend{Поиск без фильтрации и сортировки, 	Поиск с фильтрацией и сортировкой}
			\end{axis}
		\end{tikzpicture}
	\end{center}
	\caption{Зависимость времени выполнения поиска в ElasticSearch в зависимости от запрашиваемого количества записей}
	\label{fig:time-es}
\end{figure}

\clearpage

Из результатов видно, что Postgres с фильтрами дольше примерно на 140\%. Это обусловлено необходимостью применения фильтров к различным полям и необходимостью дополнительных JOIN-ов, т.к. некоторые фильтры используют данные из связанных таблиц.

Elastic с фильтрами дольше примерно на 130\%. Это обусловлено необходимостью применения фильтров к различным полям и необходимостью сортировки результатов.

Можно заметить, что в среднем Elastic работает медленнее Postgres в 4 раза. Это может быть связанно с тем, что ElasticSearch использует протокол <<HyperText Transfer Protocol>> (HTTP) \cite{http}, в то время как Postgres взаимодействует с приложением по протоколу <<Transmission Control Protocol>> (TCP), который передает данные быстрее. 

\subsection*{Вывод}
Несмотря на необходимость выборки данных из нескольких таблиц, поиск событий в Postgres оказался более эффективным, чем поиск событий в \linebreak ElasticSearch примерно в 4 раза. 

Таким образом, для реализации сервиса поиска событий оптимальнее использовать Postgres, чем ElasticSearch.







