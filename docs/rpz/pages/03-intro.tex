\section*{ВВЕДЕНИЕ}
\addcontentsline{toc}{section}{ВВЕДЕНИЕ}

Планирование является неотъемлемым процессом в современной жизни. Необходимость постоянно держать в голове большое количество встреч, мероприятий, событий и т.д. привела к появлению средств, которые позволяют хранить подобную информацию. Одним из видов таких средств является календарь, хранящий информацию о событиях. Но в одном событии может учавствовать сразу несколько человек, одного из которых можно назвать инициатором этого события, а остальных --- гостями. По этой причине появляется необходимость в создании информационной системы, позволяющей нескольким пользователям совместно планировать события.

Целью курсовой работы является разработка базы данных для многопользовательского календаря. 

Для достижения поставленной цели необходимо выполнить следующие задачи:
\begin{enumerate}[label=\arabic*)]
	\item проанализировать существующие решения;
	\item формализовать задачу и определить необходимый функционал; 
	\item рассмотреть модели баз данных и выбрать наиболее подходящую; 
	\item проанализировать существующие СУБД и выбрать наиболее оптимальные; 
	\item спроектировать и разработать базу данных;
	\item спроектировать и разработать приложение, взаимодействующее с разработанной базой данных;
	\item провести сравнительный анализ эффективности фильтрации и сортировки при использовании различных СУБД.
\end{enumerate}