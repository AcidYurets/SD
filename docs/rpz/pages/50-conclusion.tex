\section*{ЗАКЛЮЧЕНИЕ}
\addcontentsline{toc}{section}{ЗАКЛЮЧЕНИЕ}

\textbf{Поставленная цель была достигнута} --- разработана база данных для многопользовательского календаря и приложение, взаимодействующее с разработанной базой данных.

Для достижения поставленной цели были решены следующие задачи:
\begin{itemize}[label=---]
	\item проанализированы существующие решения;
	\item формализована задачу и определен необходимый функционал; 
	\item рассмотрены модели баз данных и выбрана наиболее подходящая; 
	\item проанализированы существующие СУБД и выбраны наиболее оптимальные; 
	\item спроектирована и разработана базу данных;
	\item спроектировано и разработано приложение, взаимодействующее с разработанной базой данных;
	\item проведен сравнительный анализ эффективности фильтрации и сортировки при использовании различных СУБД.
\end{itemize}

Программный продукт представляет из себя приложение с API, разработанным согласно спецификации GraphQL. Приложение позволяет создавать события с указанием названия, временной метки, описания, типа,  тегов и приглашенных пользователей. Также присутствует возможность для приглашенного пользователя просматривать, менять событие или приглашать других участников в зависимости от выданных ему прав доступа к событию. Помимо этого, реализован сервис поиска, позволяющий находить события по заданным фильтрам и предусматривающий возможность из сортировки.


Из результатов проведенного сравнительного анализа следует, что для реализации сервиса поиска событий оптимальнее использовать Postgres, чем \linebreak ElasticSearch. 


